\documentclass[a4paper,10pt]{scrartcl}
\usepackage[utf8]{inputenc}
\usepackage[hyphens]{url}
\usepackage{hyperref}
\usepackage[ngerman]{babel}
\usepackage[T1]{fontenc}
\usepackage{graphicx}
\usepackage[utf8]{inputenc}
\usepackage{lmodern}
\usepackage{geometry}
\usepackage{pgfplots} 
\usepackage{mathrsfs}
\usepackage{mathtools}
\usepackage{listings}
\usepackage{wrapfig}
\usepackage{float}
\usepackage{amsmath}
\usepackage{stmaryrd}
\usepackage{paralist}
\lstset{basicstyle=\normalfont\ttfamily,breaklines=true}


\geometry{paper=a4paper, left=20mm, right=20mm, top=30mm, bottom=30mm}
\hypersetup{
    unicode=false,
    pdftoolbar=true,
    pdfmenubar=true,
    pdffitwindow=false,
    pdfstartview={FitH},
    pdftitle={ML1 Ex4},
    pdfauthor={Patrick Bonack},
    pdfsubject={Subject},
    pdfcreator={Creator},
    pdfproducer={Producer},
    pdfkeywords={keyword1} {key2} {key3},
    pdfnewwindow=true,
    colorlinks=false,
    linkcolor=red,
    citecolor=green,
    filecolor=magenta,
    urlcolor=cyan
}

\title{\vspace{-2cm}Bericht DataSciece Projekt}
\subtitle{Verbindung von Arbeitsmarkt und Wahlverhalten in Frankfurt}
\author{PaSeDa}
\date{}

\usepackage{etoolbox}


%Mathepakete
\usepackage{amsfonts}
\usepackage{amsmath}
\usepackage{amssymb}
\usepackage{amsthm}    



\allowdisplaybreaks
\begin{document}
\maketitle

\section{define the goal}
done

\section{Take existing open data from at least two different sources}
done

\section{Merge and clean the data}
sind dabei
\section{Test and verify your data quality}
wie machen wir das?
\section{Further preprocessing}
TODO
\section{Apply two different algorithms of the same kind}
Wir haben ein interessantes Problem hier. Wir wollen weder eine Klassifizierung noch eine Regression. Unser gesuchtes Ergebnis ist die Ausprägung(Stimmanteil) verschiedener Parteien(Klassen). Ich denke One-versus-the-rest bzw. Multinomial logistic regression sollten wir auf jeden Fall anwenden.

\section{Evaluate and verify the results of them!}
TODO
\section{Come up with a conclusion!}
TODO
\end{document}
